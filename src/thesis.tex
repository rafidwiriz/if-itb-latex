%--------------------------------------------------------------------%
%
% Berkas utama templat LaTeX.
%
% author Petra Barus, Peb Ruswono Aryan
%
%--------------------------------------------------------------------%
%
% Berkas ini berisi struktur utama dokumen LaTeX yang akan dibuat.
%
%--------------------------------------------------------------------%

\documentclass[toc=listof, 12pt, a4paper, onecolumn, oneside, final]{report}

%-------------------------------------------------------------------%
%
% Konfigurasi dokumen LaTeX untuk laporan tesis IF ITB
%
% @author Petra Novandi
%
%-------------------------------------------------------------------%
%
% Berkas asli berasal dari Steven Lolong
%
%-------------------------------------------------------------------%

% Ukuran kertas
\special{papersize=210mm,297mm}

% Setting margin
\usepackage[top=3cm,bottom=2.5cm,left=4cm,right=2.5cm]{geometry}

\usepackage{mathptmx}

% Judul bahasa Indonesia
\usepackage[bahasa]{babel}

% Format citation
\usepackage[backend=bibtex,style=ieee,citestyle=numeric]{biblatex}

\usepackage[utf8]{inputenc}
\usepackage{microtype}
\usepackage{makecell}
\usepackage{graphicx}
\usepackage{tabularx}
\usepackage{listings}
\usepackage{tabto}
\usepackage{comment}
\usepackage{amsmath}
\usepackage[labelfont=bf]{caption}	% Package dengan opsi untuk mempertebal label caption
\usepackage{enumitem}				
\usepackage{tocbibind}				% Package untuk memasukkan Daftar Pustaka ke dalam Daftar Isi
\usepackage{tocloft}				% Package untuk mengatur Daftar Isi, Daftar Gambar dan Daftar Tabel
\usepackage{float}					% Package untuk membantu penetapan lokasi gambar
\usepackage{indentfirst}			% Package untuk indentasi pada paragraf pertama
\usepackage[auto]{chappg}			% Package untuk penomoran halaman Bab-Halaman
\usepackage{titling}
\usepackage{blindtext}
\usepackage{sectsty}
\usepackage{chngcntr}
\usepackage{etoolbox}
\usepackage{hyperref}       		% Package untuk link di daftar isi.
\usepackage{titlesec}       		% Package Format judul
\usepackage{parskip}
\usepackage[htt]{hyphenat}

% Line satu setengah spasi
\renewcommand{\baselinestretch}{1.5}

% Setting judul
\titlespacing*{\chapter}{0pt}{-50pt}{10pt}
\chapterfont{\centering \Large}
\titleformat{\chapter}[display]
  {\Large\centering\bfseries}
  {\chaptertitlename\ \thechapter}{0pt}
    {\Large\bfseries\uppercase}

% Setting nomor pada subbsubsubbab
\setcounter{secnumdepth}{3}
\setcounter{tocdepth}{4}

\makeatletter
\setlength{\@fptop}{0pt}
\setlength{\@fpbot}{0pt plus 1fil}
\makeatother

% Counter untuk figure dan table.
\counterwithin{figure}{chapter}
\counterwithin{table}{chapter}

% Counter untuk penomoran halaman lanjut
\newcounter{savepage}

% Pengaturan caption
\captionsetup{labelsep=space}

% Pengaturan spasi untuk justify
\pretolerance=10000
\tolerance=2000 
\emergencystretch=10pt
%\tolerance=1
%\emergencystretch=10pt
%\hyphenpenalty=10000
%\exhyphenpenalty=10000

% Pengaturan untuk Daftar Rumus
\newcommand{\listequationsname}{Daftar Rumus}
\newlistof{myequations}{equ}{\listequationsname}
\newcommand{\myequations}[1]{%
\addcontentsline{equ}{myequations}{\protect\numberline{\theequation}#1}\par}
\setlength{\cftmyequationsnumwidth}{2.5em}

\newcommand{\listalgorithmname}{Daftar Algoritme}
\newlistof{algorithm}{alg}{\listalgorithmname}
\newcommand{\algorithm}[1]{%
\refstepcounter{algorithm}
\par\noindent\centering\normalsize\textbf{Algoritme {\thechapter.\thealgorithm\ }\normalfont #1}
\addcontentsline{alg}{algorithm}{\protect\numberline{\thechapter.\thealgorithm}#1}\par}
\setlength{\cftalgorithmnumwidth}{2.5em}

% Pengaturan untuk title Daftar Isi, Tabel, dan Gambar
\renewcommand{\cfttoctitlefont}{\hspace*{\fill}\Large\bfseries\MakeUppercase}
\renewcommand{\cftaftertoctitle}{\hspace*{\fill}}
\renewcommand{\cftlottitlefont}{\hspace*{\fill}\Large\bfseries\MakeUppercase}
\renewcommand{\cftafterlottitle}{\hspace*{\fill}}
\renewcommand{\cftloftitlefont}{\hspace*{\fill}\Large\bfseries\MakeUppercase}
\renewcommand{\cftafterloftitle}{\hspace*{\fill}} 
\renewcommand{\cftequtitlefont}{\hspace*{\fill}\Large\bfseries\MakeUppercase}
\renewcommand{\cftafterequtitle}{\hspace*{\fill}} 
\renewcommand{\cftalgtitlefont}{\hspace*{\fill}\Large\bfseries\MakeUppercase}
\renewcommand{\cftafteralgtitle}{\hspace*{\fill}} 

\renewcommand{\cftchappresnum}{Bab }
\renewcommand{\cftchapaftersnum}{}
\renewcommand{\cftchapnumwidth}{3.7em}

\setlength{\cftbeforetoctitleskip}{-4em}
\setlength{\cftbeforeloftitleskip}{-4em}
\setlength{\cftbeforelottitleskip}{-4em}
\setlength{\cftbeforeequtitleskip}{-4em}
\setlength{\cftbeforealgtitleskip}{-4em}

\newcolumntype{Y}{>{\centering\arraybackslash}X}

\lstset{frame=single,
  columns=fullflexible,
  basicstyle={\small\ttfamily},
  breaklines=true,
  breakatwhitespace=false,
  postbreak=\mbox{$\hookrightarrow$\space},
  tabsize=3
}

\renewcommand{\lstlistingname}{Algoritme}
\renewcommand{\lstlistlistingname}{Daftar \lstlistingname}

\renewcommand{\cftchapleader}{\cftdotfill{\cftdotsep}}

\cftsetpnumwidth{2em}

\DefineBibliographyStrings{english}{%
    urlseen = {Waktu akses},
    url = {URL:},
    and = {dan},
    andothers = {dkk\adddot},
    in = {dalam}
}

\makeatletter

\makeatother

\bibliography{references}

\begin{document}

    %Basic configuration
    \title{Normalisasi Kata Tidak Baku yang Tidak Disingkat dengan Jarak Perubahan}
    \date{26 Februari}
    \newcommand{\yearsidang}{2019}
    \author{Rafi Dwi Rizqullah}
    \newcommand{\nim}{18214035}

    \pagenumbering{roman}
    \setcounter{page}{0}

    \clearpage
\pagestyle{empty}

\begin{center}
\smallskip

    \Large \bfseries \MakeUppercase{\thetitle}
    \vfill

    \Large Laporan Tugas Akhir
    \vfill

    \large Disusun sebagai syarat kelulusan tingkat sarjana
    \vfill

    \large Oleh

    \Large \theauthor

    \vfill
    \begin{figure}[h]
        \centering
      	\includegraphics[width=0.15\textwidth]{resources/cover-ganesha.jpg}
    \end{figure}
    \vfill

    \large
    \uppercase{
        Program Studi Sistem dan Teknologi Informasi \\
        Sekolah Teknik Elektro dan Informatika \\
        Institut Teknologi Bandung
    }

    2018

\end{center}

\clearpage

    \clearpage
\pagestyle{empty}

\begin{center}
\smallskip

    {\Large \bfseries Lembar Pengesahan}
    \addcontentsline{toc}{chapter}{Lembar Pengesahan}

    \MakeUppercase{\normalsize \bfseries \thetitle}
    \vfill

    \normalsize Tugas Akhir \\
    Program Studi: Sarjana Sistem dan Teknologi Informasi \\
    Sekolah Teknik Elektro dan Informatika \\
    Institut Teknologi Bandung \\
    \vfill

    \normalsize oleh :

    \normalsize \theauthor \\
    \normalsize NIM: \nim

    \vfill
    \normalsize \normalfont
    Telah disetujui dan disahkan sebagai Laporan Tugas Akhir \\
    di Bandung, pada tanggal \thedate{} \yearsidang{}.

    \vfill
    \setlength{\tabcolsep}{12pt}
    \begin{tabularx}{\textwidth}{c@{\hskip 0.2\textwidth}cc@{\hskip 0.3\textwidth}}
        & {\bfseries Pembimbing} & \\
        & & \\
        & & \\
        & & \\
        & & \\
        & \underline{I Gusti Bagus Baskara Nugraha ST,MT,Ph.D.} & \\
        & NIP. 19760124 201012 1 001 & 
    \end{tabularx}

\end{center}
\clearpage

    \chapter*{Lembar Pernyataan Orisinalitas}
\vspace{15mm}

Dengan ini saya menyatakan, Tugas Akhir ini adalah hasil karya sendiri dan semua referensi telah diacu dengan benar sesuai dengan kaidah penulisan ilmiah.
\vspace{15mm}

Bandung, 5 September 2018 \\
\\
\\
\\
\\
\theauthor


    \pagestyle{plain}

    \clearpage
\chapter*{ABSTRAK}
\addcontentsline{toc}{chapter}{Abstrak}

%taruh abstrak bahasa indonesia di sini
Teknologi \textit{voice assistant} mulai berkembang pesat saat ini. Penggunaannya sudah mulai merambah kepada penggunaan sehari-hari. Namun, \textit{voice assistant} masih terbatas pada penggunaan bahasa percakapan yang baku. Laporan ini akan membahas mengenai penelitian mengenai pemakaian model latihan yang cocok digunakan untuk penggunaan Bahasa Indonesia sehari-hari. Penelitian diawali dengan analisis kelemahan dari produk yang terkait dengan \textit{voice assistant}. Lalu, analisis tersebut digunakan untuk merancang model latihan yang cocok untuk kasus ini.

\vspace{15mm}
Kata kunci: \textit{voice assistant}, percakapan, model latihan.
\clearpage
    \clearpage
\chapter*{ABSTRACT}
\addcontentsline{toc}{chapter}{Abstract}

%put your abstract here
Voice assistant technology is growing rapidly now. Its use has begun to spread to daily use. However, voice assistants are still limited to the use of standard conversation languages. Meanwhile, Indonesian people are accustomed to saying non-formal language in everyday conversation. The execution of this Final Project includes solutions to overcome the problem of voice assistants with non-formal words or not included in the formal word dictionary. The approach used as a solution is to normalize the text using Levenshtein distance and Jaro-Winkler distance. Test result shows that normalization using Levenshtein distance outperform the normalization using LCS distance with accuracy difference of 8.34 percent.

\vspace{15mm}
Key words: voice assistant, dictionary, non-formal word, normalization, Levenshtein distance, Jaro-Winkler distance.

\clearpage
    \chapter*{Kata Pengantar}
\addcontentsline{toc}{chapter}{Kata Pengantar}

Puji Syukur penulis panjatkan kehadirat Allah SWT, karena atas kehendak dan rahmat-Nya penulis dapat menyelesaikan penulisan Laporan Tugas Akhir dengan judul "\thetitle " untuk memenuhi syarat dari mata kuliah Tugas Akhir serta kelulusan tingkat sarjana dari program studi Sistem dan Teknologi Informasi Institut Teknologi Bandung.

Penulis mengucapkan terima kasih kepada Bapak Dr. I Gusti Bagus Baskara Nugraha S.T., M.T. selaku dosen STEI ITB yang telah bersedia membimbing penulis dalam pelaksanaan Tugas Akhir dan pengerjaan laporan ini. Penulis mendapatkan pelajaran yang berharga dalam mengerjakan Tugas Akhir dan menulis laporan ini yang dapat dimanfaatkan untuk kedepannya.

Penulis mengucapkan terima kasih kepada keluarga penulis, kedua orangtua dan seorang kakak, yang selalu mendukung penulis untuk menjalani perkuliahan serta Tugas Akhir ini. Kasih sayang dari mereka telah membawa penulis sampai pada titik ini.

Penulis juga berterima kasih kepada pihak PT. Zamrud Technology yang telah bersedia membantu penulis dalam pengerjaan Tugas Akhir ini. Penulis juga dapat mempelajari ilmu baru mengenai pengerjaan Tugas Akhir penulis selama penulis berada di sana.

Terakhir, penulis mengucapkan terima kasih kepada teman-teman penulis yang telah mendukung penulis hingga penulis dapat menyelesaikan Tugas Akhir ini. Meskipun tidak membantu secara langsung, penulis merasa terbantu oleh teman-teman penulis dalam hal motivasi untuk melaksanakan Tugas Akhir ini.

Demikian pesan terima kasih dari penulis. Penulis benar-benar bersyukur dengan semua orang-orang yang telah mendukung penulis dalam pelaksanaan Tugas Akhir ini. Sekian, terima kasih.

\vspace{15mm}
\begin{tabularx}{\textwidth}{l@{\hskip 0.5\textwidth}l}
    & Bandung, \thedate{} \yearsidang{}\\
    & Penulis
\end{tabularx}

    \titleformat*{\section}{\centering\bfseries\Large\MakeUpperCase}
    
	\clearpage
    \tableofcontents
    
    \clearpage
    {%
		\let\oldnumberline\numberline%
		\renewcommand{\numberline}{\figurename~\oldnumberline}%
		\listoffigures%
	  }
	
	\clearpage
    {%
		\let\oldnumberline\numberline%
		\renewcommand{\numberline}{\tablename~\oldnumberline}%
		\listoftables%
    }

    \clearpage
    {%
		\let\oldnumberline\numberline%
        \renewcommand{\numberline}{Rumus~\oldnumberline}%
        \listofmyequations%
        \addcontentsline{toc}{chapter}{Daftar Rumus}
    }

    \clearpage
    {%
		\let\oldnumberline\numberline%
        \renewcommand{\numberline}{Algoritme~\oldnumberline}%
        \lstlistoflistings%
    }
    %\chapter*{Daftar Singkatan}

\begin{table}[H]
    \centering
	\begin{tabularx}{\textwidth}{cXc}
        \textbf{Singkatan} & \textbf{Kepanjangan dari} & \textbf{Halaman Kemunculan Pertama} \\ \hline
        SLU & \textit{Spoken Language Understanding} & \pageref{abb:slu} \\
        CNN & \textit{Convolutional Neural Network}  & \pageref{abb:cnn}
	\end{tabularx}
\end{table}
    \newpage
    \setcounter{savepage}{\arabic{page}}

    \titleformat*{\section}{\bfseries\Large}

    %----------------------------------------------------------------%
    % Konfigurasi Bab
    %----------------------------------------------------------------%    
    \renewcommand{\chaptername}{BAB}
    \renewcommand{\thechapter}{\Roman{chapter}}
    \pagenumbering{bychapter}
    \setlength{\parindent}{1cm}
    %----------------------------------------------------------------%

    %----------------------------------------------------------------%
    % Dafter Bab
    % Untuk menambahkan daftar bab, buat berkas bab misalnya `chapter-6` di direktori `chapters`, dan masukkan ke sini.
    %----------------------------------------------------------------%
    \chapter{Pendahuluan}

\section{Latar Belakang}

\textit{Voice assistant} mulai menjadi tren yang makin berkembang. Jejaknya mulai terlihat belakangan ini dalam kehidupan manusia. Gartner menyebutkan, pada tahun 2018, 30 persen interaksi dengan sebuah teknologi melibatkan media yang mendukung percakapan dengan mesin pintar \parencite{escherich2015market}. Tren tersebut diperkirakan akan terus berlanjut hingga beberapa tahun kedepan. ComScore memperkirakan bahwa tahun 2020, 50 persen pencarian dalam internet merupakan pencarian berbasis suara.

Berbicara dengan aplikasi \textit{voice assistant} dapat menciptakan pengalaman yang lebih baik untuk pengguna. 41 persen pengguna \textit{voice assistant} merasa bahwa berbicara dengan \textit{voice assistant} terasa seperti berbicara dengan teman atau orang lain \parencite{kleinberg2018five}. Dan data tersebut diperkuat dengan data bahwa hampir 70 persen pencarian dengan menggunakan Google Assistant menggunakan bahasa alami ketimbang bahasa terstruktur yang biasa diguakan pada pencarian web pada umumnya. Pengalaman tersebut menjadikan pengguna merasa nyaman menggunakan \textit{voice assistant} dalam kegiatan sehari-hari. Data dari Google menunjukkan 72 persen menggunakan aplikasi \textit{voice assistant} sebagai bagian dari rutinitas harian mereka \parencite{kleinberg2018five}.

Dengan keuntungan yang bagus, \textit{voice assistant} memiliki potensi untuk mempermudah kegiatan bagi masyarakat Indonesia. Namun, untuk dapat diterima oleh masyarakat, \textit{voice assistant} harus mengatasi sebuah keadaan yang mana hampir 20 persen dari seluruh masyarakat Indonesia yang berbicara dengan menggunakan bahasa Indonesia untuk percakapan sehari-hari \parencite{naim2012kewarganegaraan}. Masayarakat Indonesia terbiasa menggunakan bahasa selain bahasa Indonesia untuk berbicara dengan kerabat mereka dalam lingkungan tidak resmi, lebih banyak menggunakan bahasa Indonesia dengan campur kode. Campur kode menurut Chaer dan Agustina \parencite{chaer1995sosiolinguistik} adalah pemakaian dua bahasa atau lebih atau dua varian dari sebuah bahasa dalam satu masyarakat tutur, dimana salah satu merupakan kode utama atau kode dasar yang digunakan yang memiliki fungsi dan keotomiannya, sedangkan kode-kode lain yang terlibat dalam peristiwa tutur hanya berupa serpihan-serpihan saja. Dengan begitu, \textit{voice assistant} harus dapat mengerti bahasa campur kode tersebut dalam hal menambah pengalaman untuk para pengguna berupa masyarakat Indonesia.

Namun, masalah terjadi pada model bahasa untuk \textit{voice assistant}. Masalah yang terjadi adalah terdapat kata-kata campur kode yang tidak termasuk ke dalam kamus bahasa Indonesia yang baku. Sebagai contoh, kata "\textit{puter}" yang biasa diartikan sebagai kata "putar", "\textit{maen}" yang diartikan sebagai "main", dan lain-lain. Selain itu, campur kode juga melibatkan kata-kata yang berasal dari bahasa selain bahasa Indonesia, termasuk bahasa daerah dan bahasa asing. Padahal, model bahasa yang sudah tersedia menggunakan media-media resmi sebagai acuan untuk membentuk model bahasanya. Sebagai contoh, model bahasa fastText milik Facebook dibangun dan dilatih dengan acuan kepada kata-kata dalam Wikipedia, termasuk untuk bahasa Indonesia \parencite{bojanowski2016enriching}. Model bahasa fastText yang telah dilatih terdiri dari kamus dan nilai vektor untuk tiap kata di dalam kamus tersebut. Nilai vektor tiap kata berguna untuk "meletakkan" kata dalam sebuah "tempat" multi dimensi dan menentukan kata yang saling berdekatan dengan kata yang lain. Dengan model yang sudah dilatih tersebut, menambah beberapa kata ke dalam model bahasa tidak dimungkinkan.

Untuk mengatasi masalah ketiadaan kata dalam kamus, diperlukan sebuah metode berupa normalisasi kata, yaitu mengubah kata yang tidak baku menjadi kata baku dalam suatu bahasa. Terdapat sebuah penelitian mengenai pembuatan sistem normalisasi teks pada suntingan Twitter bahasa Indonesia dengan menggunakan bahasa pemrograman R \parencite{saragih2017normalisasi}. Sistem tersebut mengukur jarak perbedaan kedua kata atau \textit{string} dengan menggunakan fungsi \textit{stringdist} dengan 10 variasi metode penghitungan jarak yang sudah disediakan. Namun, sistem tersebut memiliki beberapa kelemahan, seperti sistem harus mendefinisikan terlebih dahulu kamus kata-kata yang tidak baku beserta perbaikan kata bakunya secara manual, termasuk kata tidak baku yang hanya berjarak beberapa huruf dengan kata baku yang mirip. Pendefinisian tersebut menambah beban pekerjaan pengurus sistem, terlebih jika terdapat kata tidak baku yang baru. Lalu, kamus perbaikan kata tidak baku tersebut diletakkan sebelum pencarian dalam kamus kata baku menggunakan \textit{stringdist} sehingga peran terbesar dalam sistem tersebut ada pada kamus perbaikan kata tidak baku.

Untuk menghindari penambahan kata kamus untuk normalisasi, dibutuhkan pendekatan berupa normalisasi dengan menggunakan kamus yang sudah tersedia, kemudian membuat nilai vektor untuk tiap kata dalam kamus dengan pelatihan sehingga sistem \textit{voice assistant} dapat memperkirakan kata masukan yang tidak terdapat dalam kamus. Selain itu, diperlukan penanganan kasus berupa kata yang sama sekali tidak memiliki kemiripan dengan kata yang ada di dalam kamus.

\section{Rumusan Masalah}

Dari latar belakang yang telah dijabarkan, masalah yang akan diselesaikan dalam pengerjaan Tugas Akhir ini adalah "bagaimana cara membangun fitur normalisasi untuk mengatasi kelemahan sistem yang tidak dapat memperkirakan kata yang tidak ada dalam kamus?"

\section{Tujuan}

Berdasarkan masalah yang telah dirumuskan sebelumnya, Tugas Akhir ini dikerjakan dengan tujuan untuk membangun fitur normalisasi untuk mengatasi kelemahan sistem yang tidak dapat memperkirakan kata yang tidak ada dalam kamus.

\section{Batasan Masalah}

Batasan masalah dalam pengerjaan Tugas Akhir ini hanya terletak pada bagian klasifikasi teks. Pengerjaan Tugas Akhir tidak terlibat pada sistem suara menjadi teks. Selain itu, batasan masalah juga terletak pada kata-kata yang diucapkan secara lisan saja sehingga tidak ada kata yang disingkat.

\section{Metodologi}

Metodologi yang digunakan dalam perancangan fitur nromalisasi sistem \textit{voice assistant} dalam pengerjaan Tugas Akhir ini adalah sebagai berikut:

\begin{enumerate}
	\item melakukan analisis terkait kebutuhan sistem \textit{voice assistant},
	\item melakukan studi literatur mengenai metode-metode yang akan digunakan untuk fitur normalisasi,
	\item melakukan pengembangan sistem untuk mengimplementasi fitur dengan metode yang telah ditentukan, dan,
	\item melakukan evaluasi sistem setelah fitur diimplementasi. 
\end{enumerate}

\section{Sistematika Pembahasan}

Laporan pengerjaan Tugas Akhir ini disusun dengan sistematika sebagai berikut:

\begin{enumerate}[label=Bab \arabic*,itemindent=*]
	\item Pendahuluan\\
	Bab ini membahas latar belakang, tujuan, batasan masalah, metodologi, dan sistematika penulisan.
	\item Dasar Teori\\
	Bab ini membahas teori yang digunakan untuk menyelesaikan rumusan masalah.
	\item Metodologi dan Rancangan\\
	Bab ini membahas metodologi pengerjaan Tugas Akhir dan rancangan yang diajukan untuk mengatasi rumusan masalah.
	\item Implementasi dan Pengujian\\
	Bab ini membahas implementasi dan hasil pengujian rancangan baru yang telah dibangun beserta penjelasan spesifikasi lingkungan yang digunakan.
	\item Penutup\\
	Bab ini berisi kesimpulan dari rancangan yang telah diuji serta saran yang dapat digunakan untuk pengembangan kedepan.
\end{enumerate}

    \chapter{Tinjauan Pustaka}

\section{Dasar Teori}

\subsection{\textit{Natural Language Understanding} (NLU)}

Pemahaman bahasa alami, atau lebih dikenal dengan \textit{natural language understanding} (NLU), merupakan salah satu bagian dari pemrosesan bahasa alami atau \textit{natural language processing} (NLP). NLU adalah ranah dalam linguistik komputasional yang didedikasikan untuk memahami bahasa alami \parencite{harris2004voice}. Menurut Gartner, NLU diartikan sebagai pemahaman komputer terhadap struktur dan makna dari bahasa manusia, sehingga pengguna dapat berinteraksi dengan komputer menggunakan bahasa yang digunakan oleh pengguna.

Dasar dari NLU berasal dari enam bentuk pengetahuan yang diketahui dan dipelajari oleh manusia pada umumnya. Bentuk-bentuk pengetahuan tersebut didefinisikan sebagai berikut: \parencite{allen1995natural}

\begin{enumerate}
	\item pengetahuan fonetik dan fonologi, menyangkut pada bagaimana suara diubah menjadi susunan kata-kata. Dalam ilmu komputer, pengetahuan ini diatasi dengan menggunakan pengenalan suara (\textit{speech recognition}),
	\item pengetahuan morfologi, menyangkut pada bagaimana sebuah kata disusun dari beberapa morfem, seperti kata “terabaikan” disusun oleh tiga morfem, yaitu “ter-”, “abai”, dan “-kan”,
	\item pengetahuan sintaktis, menyangkut pada bagaimana kata-kata dapat disusun sehingga menjadi sebuah kalimat yang benar menurut tata bahasa,
	\item pengetahuan semantik, menyangkut bagaimana makna dari kata-kata disusun untuk membentuk makna dari sebuah kalimat,
	\item pengetahuan pragmatik, menyangkut bagaimana sebuah kalimat diartikan jika digunakan dalam konteks yang berbeda, dan,
	\item pengetahuan dunia, menyangkut memahami pengetahuan umum yang dipahami juga oleh pengguna untuk menjaga percakapan berjalan dengan semestinya.
\end{enumerate}

Pengetahuan yang termasuk ke dalam NLU adalah pengetahuan sintaksis, semantik, dan pragmatik.

\subsection{\textit{Spoken Language Understanding} (SLU)}

Pemahaman bahasa ucapan, atau \textit{spoken language understanding} (SLU), merupakan ranah yang berada di antara \textit{speech recognition}, NLP yang memanfaatkan pembelajaran mesin (\textit{machine learning}) dan kecerdasan buatan (\textit{artificial intelligence}) \parencite{tur2011spoken}. Biasanya, SLU hanya terlibat dalam dua tugas utama, yaitu klasifikasi maksud kalimat (\textit{intent classification}) dan pengisian slot (\textit{slot filling}) \parencite{goo2018slot}. Karena kedua tugas utama tersebut, SLU digunakan dalam sistem yang membutuhkan satu kalimat masukan saja dan tidak terlalu panjang.

Gambar \ref{fig:slu_early} menunjukkan rancangan sistem SLU paling awal. Sistem hanya terdiri dari \textit{automatic speech recognition} (ASR) dan NLU. Tiap komponen memiliki sumber pengetahuan masing-masing, ASR memiliki ASR KS dan NLU memiliki NLU KS. Tiap sumber pengetahuan dialirkan menuju kendali pada masing-masing proses sebelum digunakan pada proses tersebut.

\begin{figure}[H]
	\centering
	\includegraphics[width=0.5\textwidth, trim=2 2 2 2, clip]{resources/2/early_slu.pdf}
	\caption{Rancangan awal sistem SLU \parencite{tur2011spoken}}
	\label{fig:slu_early}
\end{figure}

Klasifikasi maksud kalimat berarti menentukan sebuah makna dari sebuah kalimat yang diberikan oleh pengguna. Sebagai contoh, kalimat “putarkan sebuah lagu” memiliki makna “putar media” berupa lagu. Makna tersebut kemudian diterjemahkan menjadi sebuah aksi oleh sistem. Makna kalimat bisa didapatkan dengan melihat kata-kata yang terkandung di dalam sebuah kalimat, atau melihat pola urutan kata dalam kalimat tersebut.

Pengisian slot adalah metode untuk mengambil entitas-entitas dari sebuah kalimat biasa. Beberapa aksi terkadang membutuhkan masukan parameter untuk melengkapi aksi tersebut. Masukan parameter didapatkan dari dalam kalimat yang dimasukkan oleh pengguna. Sebagai contoh, “putar lagu separuh aku dari noah” berarti pengguna menginginkan sebuah lagu diputarkan oleh sistem. Namun, permintaan pengguna menjadi lebih spesifik karena pengguna menyebutkan judul lagu beserta artis. Judul lagu adalah “separuh aku” dengan artis adalah “noah”. Metode yang dapat digunakan untuk mengambil entitas adalah melakukan pelabelan kata-kata yang menjadi entitas dengan menggunakan \textit{recurrent neural network} (RNN).

\subsection{\textit{Machine Learning}}

Pembelajaran mesin, atau \textit{machine learning}, adalah sebuah program komputer yang dirancang untuk belajar dari pengalaman dengan acuan berupa kelas tugas-tugas dan pengukuran performa, jika performa pada suatu tugas, yang telah diukur, diperbaiki dengan pengalaman \parencite{mitchell1997machine}. Pembelajaran mesin menggunakan data-data yang telah terkumpul untuk membentuk sebuah pola yang dapat digunakan pada data yang akan muncul atau dimasukkan oleh pengguna.

Pembelajaran mesin diterapkan pada berbagai macam aplikasi, yang ditujukan untuk mengurangi peran manusia dalam melakukan hal tersebut. Sebagai contoh, pembelajaran mesin digunakan untuk melakukan klasifikasi gambar, teks, suara, dan lain-lain. Selain itu, pembelajaran mesin diaplikasikan ke dalam analisis data dengan jumlah yang sangat besar atau biasa disebut sebagai \textit{big data}.

\subsection{\textit{Artificial Neural Network} (ANN)}

\textit{Artificial neural network} (ANN) adalah sistem komputasi yang memiliki elemen-elemen pemrosesan sederhana yang saling terhubung, yang mana dapat memproses informasi dengan respon keadaan dinamisnya kepada masukan dari luar \parencite{caudill1987neural}. ANN terdiri dari tiga jenis lapisan, yaitu lapisan masukan, lapisan tersembunyi, serta lapisan keluaran. Lapisan masukan adalah lapisan yang berisi nilai-nilai masukan berasal dari data yang telah terkumpul. Lapisan tersembunyi berisi fungsi-fungsi aktivasi yang digunakan untuk menentukan keluaran yang diinginkan oleh \textit{neural network} secara keseluruhan.

\textit{Neural network} memiliki banyak variasi untuk kondisi data yang berbeda-beda. Jenis \textit{neural network} yang sangat dasar adalah \textit{feed forward neural network}, digunakan untuk melakukan klasifikasi sederhana. Untuk laporan ini, jenis \textit{neural network} yang digunakan adalah \textit{neural network} yang dapat digunakan dalam melakukan klasifikasi sekuensial, seperti \textit{recurrent neural network} (RNN) dan \textit{convolutional neural network} (CNN).

\subsection{\textit{Convolutional Neural Network} (CNN)}

\textit{Convolutional neural network} atau CNN biasa digunakan untuk mengenali objek-objek yang berada di dalam sebuah gambar. CNN mencoba memprediksi sebuah objek yang bersangkutan dengan mendeteksi fitur-fitur yang terdapat dalam sebuah gambar menggunakan \textit{filter}. Untuk penerapan di dalam teks, fitur-fitur tersebut didapatkan dengan melakukan \textit{embedding} pada sebuah kata yang bersangkutan. \textit{Embedding} adalah proses untuk merepresentasikan sebuah kata menjadi vektor multi dimensi.

Keuntungan yang dimiliki oleh CNN, jika dibandingkan dengan \textit{recurrent neural network} (RNN), adalah kemampuan CNN dalam melihat kata yang berada di depan satu kata yang sedang dilatih. Dengan begitu, CNN dapat mempertimbangkan kata sebelum dan kata selanjutnya untuk memasangkan label pada sebuah kata.

\subsection{\textit{Long Short Term Memory} (LSTM)}

\textit{Long short term memory} atau LSTM digunakan untuk mengatasi permasalahan latihan dengan menggunakan data yang bersifat \textit{sequential}. LSTM digunakan untuk mengatasi kelemahan yang dimiliki oleh RNN, yaitu RNN tidak mampu menyimpan informasi dalam jangka waktu yang sangat lama. LSTM dapat memutuskan apakah sebuah informasi akan diteruskan pada iterasi berikutnya atau dilupakan dengan bantuan dari gerbang pelupa (\textit{forget gate}).

Gambar \ref{fig:lstm} menunjukkan isi dari sebuah sel LSTM. Sel LSTM terdiri dari empat jenis gerbang, yaitu dua gerbang masukan, gerbang keluaran, dan gerbang pelupa. Semua gerbang mengandalkan masukan yang berasal dari masukan kata iterasi saat ini dan luaran dari iterasi sebelumnya. Kedua gerbang masuk mengolah nilai luaran dan masukan dengan menggunakan fungsi aktivasi sigmoid dan tanh, kemudian dilakukan operasi perkalian matriks pada kedua hasil tersebut. Gerbang pelupa mengolah kedua nilai dengan menggunakan fungsi aktivasi sigmoid. Terakhir, gerbang luaran mengolah kedua nilai dengan menggunakan fungsi aktivasi sigmoid. Kemudian terdapat sebuah nilai yang disebut dengan nilai keadaan sel. Nilai keadaan sel saat ini didapatkan dengan mengalikan nilai keadaan sel sebelumnya dengan nilai luaran gerbang pelupa, lalu ditambahkan dengan hasil dari kedua gerbang masukan.

\begin{figure}[H]
	\centering
	\includegraphics[width=0.8\textwidth, trim=2 2 2 2, clip]{resources/2/lstm.pdf}
	\caption{Arsitektur sel LSTM}
	\label{fig:lstm}
\end{figure}

\section{Analisis Produk}

Produk terkait yang menjadi acuan dalam pengerjaan Tugas Akhir adalah Rasa NLU, produk NLU \textit{open source} yang dibangun oleh Rasa HQ.

\subsection{Alat-Alat yang Digunakan}

Rasa NLU merupakan salah satu bagian dari Rasa Stack, \textit{library} Python yang digunakan untuk membangun sistem \textit{chatbot} berbasis NLU. Rasa NLU membutuhkan komponen-komponen sebagai berikut: spaCy sebagai \textit{library} pembantu Rasa NLU dalam melakukan pengolahan teks, dan fastText sebagai model vektor kata yang digunakan untuk membantu spaCy dalam mengolah teks.

\subsubsection{Rasa}

Rasa, terdiri dari Rasa NLU dan Rasa Core, adalah sepasang \textit{open-source library} Python yang digunakan untuk membangun perangkat lunak berbasis percakapan \parencite{bocklisch2017rasa}. Rasa biasanya digunakan untuk pengembangan robot percakapan atau dikenal dengan \textit{chatbot} berbentuk teks. Tujuan dari Rasa adalah membantu para pengembang untuk mengembangkan manajemen dialog dan pemahaman bahasa berdasarkan pembelajaran mesin. Pengembangan Rasa terinspirasi dari beberapa \textit{library} yang telah ada, seperti scikit-learn dan Keras untuk API, fastText untuk klasifikasi teks, dan GloVe untuk penyediaan data latih untuk \textit{word embedding}. Versi Rasa yang digunakan dalam pengerjaan Tugas Akhir, yaitu Rasa NLU versi 0.11.4 dan Rasa Core versi 0.8.6.

Arsitektur Rasa dapat dilihat pada Gambar \ref{fig:rasa_arch}. Terdapat empat bagian di dalam arsitektur Rasa, yaitu \textit{Interpreter, Tracker, Policy}, dan \textit{Action}. Bagian Interpreter ditangani oleh Rasa NLU, sedangkan bagian yang lain ditangani oleh Rasa Core. Rasa bersifat modular, sehingga \textit{library} dapat diintegrasikan dengan \textit{library} lain, seperti Rasa Core dapat dihubungkan dengan \textit{Interpreter} selain Rasa NLU, dan sebaliknya.

Tahap-tahap proses yang ada di dalam arsitektur Rasa dapat dijelaskan sebagai berikut. Pertama, pesan yang dimasukkan oleh pengguna dikirimkan ke \textit{Interpreter}. Di dalam \textit{Interpreter} dilakukan proses untuk mengekstrasi \textit{intent}, entities, dan beberapa informasi terstruktur lain dari pesan yang telah dimasukkan. Kedua, luaran dari \textit{Interpreter} diteruskan menuju \textit{Tracker}. \textit{Tracker} akan melacak \textit{state} dari percakapan dan memberikan pemberitahuan baru bahwa pesan telah diterima oleh \textit{Tracker}. Ketiga, \textit{Policy} menerima  \textit{state} dari \textit{Tracker}. Keempat, \textit{Policy} menentukan \textit{Action} yang harus dilakukan sebagai tanggapan pesan dari pengguna.Kelima, \textit{Action} yang terpilih dicatat oleh \textit{Tracker}. Terakhir, \textit{Action} mulai dieksekusi kepada pengguna, baik berupa pencarian atau mengeluarkan pesan.

\begin{figure}[H]
	\centering
	\includegraphics[width=0.9\textwidth, trim=2 2 2 2, clip]{resources/2/rasa_arch.pdf}
	\caption{Bagan arsitektur Rasa \parencite{bocklisch2017rasa}}
	\label{fig:rasa_arch}
\end{figure}

Secara sederhana, cara Rasa bekerja ditunjukkan oleh bagan pada Gambar \ref{fig:rasa_process}. Terdapat tiga proses utama dari Rasa NLU dan Rasa Core, yaitu melatih NLU, melatih dialog, dan menjalankan agen dialog.

\begin{figure}[H]
	\centering
	\includegraphics[width=0.8\textwidth, trim=2 2 2 2, clip]{resources/2/rasa_process.pdf}
	\caption{Bagan proses utama Rasa}
	\label{fig:rasa_process}
\end{figure}

\begin{enumerate}[label=\textit{\Alph*)}, itemindent=*, series=rasa_process_list]
	\item \textit{Melatih NLU}
\end{enumerate}

Pelatihan NLU dilakukan oleh Rasa NLU dengan bantuan pengolah teks seperti spaCy dan MITIE. Pelatihan NLU berguna untuk melatih sistem untuk dapat memahami teks dan entitas didalamnya yang akan dimasukkan oleh pengguna dan menyimpan model latihan untuk digunakan saat agen dialog siap untuk dijalankan.

Untuk dapat menjalankan proses ini, dibutuhkan dua data masing-masing dalam berbentuk file JSON, yaitu data latihan dan data konfigurasi. Data latihan berisi teks masukan yang dijadikan contoh, maksud dari teks tersebut, serta entitas yang terkandung di dalamnya. Sedangkan data konfigurasi berisikan konfigurasi yang digunakan sebagai acuan dalam melatih NLU, seperti bahasa, pipeline yang digunakan untuk latihan, lokasi data latihan, dan lain-lain.

Proses-proses untuk melatih NLU dapat dilihat pada Gambar \ref{fig:rasaNLU_train}, dan penjelasan tiap proses adalah sebagai berikut:

\begin{enumerate}
	\item Rasa NLU mengambil data konfigurasi, setelah itu mengambil data latihan,
	\item memuat data latihan dan mengubahnya menjadi obyek TrainingData,
	\item memuat data konfigurasi dan mengubahnya menjadi obyek RasaNLUConfig,
	\item membuat obyek Trainer sebagai “pelatih” dan mempersiapkan obyek tersebut dengan menggunakan obyek RasaNLUConfig sebagai masukan,
	\item memanggil prosedur latihan pada obyek Trainer dengan menggunakan obyek TrainingData sebagai masukan, dan,
	\item menyimpan model hasil latihan menuju direktori sistem.
\end{enumerate}

\begin{figure}[H]
	\centering
	\includegraphics[width=0.8\textwidth, trim=2 2 2 2, clip]{resources/2/rasaNLU_train.pdf}
	\caption{Bagan alur latihan NLU dari Rasa NLU}
	\label{fig:rasaNLU_train}
\end{figure}

Di dalam data konfigurasi, terdapat pengaturan \textit{pipeline} yang digunakan untuk mengatur data latihan akan dilatih dengan menggunakan komponen apa saja. Alur persiapan \textit{pipeline} yang digunakan adalah, pertama menginisiasi semua komponen, lalu melakukan latihan jika terdapat prosedur latihan didalamnya, dan terakhir menyimpan hasil latihan per komponen ke dalam direktori sistem. Dalam pengerjaan Tugas Akhir ini, \textit{pipeline template} yang digunakan adalah spacy\_sklearn yang berisikan komponen latih sebagai berikut:

\begin{enumerate}
	\item nlp\_spacy, digunakan untuk inisialisasi spaCy sebelum menggunakan komponen latih spaCy yang lain,
	\item tokenizer\_spacy, digunakan untuk melakukan tokenisasi dari teks masukan dengan menggunakan tokenisasi dari spaCy,
	\item intent\_featurizer\_spacy, digunakan untuk membuat definisi ekstrasi fitur dari teks masukan dengan menggunakan spaCy. Fitur tersebut akan digunakan untuk melakukan klasifikasi maksud kalimat,
	\item intent\_entity\_featurizer\_regex, digunakan untuk mencatatkan seluruh \textit{regular expressio}n yang telah didefinisikan dalam data latihan,
	\item ner\_crf, digunakan untuk melakukan pengenalan entitas dengan menggunakan metode latihan \textit{conditional random field} (CRF) yang disediakan oleh \textit{library} sklearn-crfsuite,
	\item ner\_synonym, digunakan untuk menampung teks-teks yang memiliki nilai entitas yang sama, dan,
	\item intent\_classifier\_sklearn, digunakan untuk melakukan latihan klasifikasi maksud kalimat dengan masukan berupa fitur teks yang telah dilakukan oleh intent\_featurizer\_spacy. Latihan klasifikasi dilakukan dengan menggunakan \textit{support vector machine} (SVM) dan pengaturan parameter latih menggunakan metode \textit{grid-search} yang disediakan oleh scikit-learn.
\end{enumerate}

\begin{enumerate}[resume*=rasa_process_list]
	\item \textit{Melatih Dialog}
\end{enumerate}

Pelatihan dialog dilakukan dengan menggunakan Rasa Core. Pelatihan dialog berguna untuk mendefinisikan domain NLU dari sistem serta melatih agen dialog untuk menciptakan prediksi respon yang tepat berdasarkan masukan dari pengguna sebelumnya. Prediksi didapatkan dengan menggunakan sebuah komponen yaitu \textit{policy}.

Untuk dapat menjalankan proses ini, dibutuhkan dua data masing-masing dalam bentuk file \textit{markdown} dan YML, yaitu data cerita dan data domain. Data cerita digunakan untuk mengkonstruksi alur percakapan yang mungkin terjadi antara pengguna dengan sistem. Data cerita terdiri dari balok-balok cerita, berisi maksud kalimat yang diekstraksi dari teks masukan pengguna dan respon sistem terhadap masukan tersebut. Jumlah respon dan timbal balik dalam satu balok cerita tidak terbatas. Data domain digunakan untuk inisialisasi definisi dari maksud kalimat, entitas yang terlibat, slot yang tersedia, serta aksi yang akan dilakukan oleh sistem.

Proses-proses untuk melatih agen dialog dijelaskan sebagai berikut:

\begin{enumerate}
	\item Rasa Core mengambil data domain dan data cerita,
	\item membuat agen dialog dengan membuat obyek Agent menggunakan masukan data domain dan policy yang digunakan,
	\item memanggil prosedur latih dari obyek Agent dengan menyertakan data cerita sebagai masukan. Prosedur ini juga membuat obyek PolicyTrainer dan melakukan pelatihan, dan,
	\item menyimpan agen dialog terlatih ke dalam direktori sistem.
\end{enumerate}

\begin{enumerate}[resume*=rasa_process_list]
	\item \textit{Menjalankan Agen Dialog}
\end{enumerate}

Setelah model NLU dan agen dialog telah tersimpan ke dalam direktori, sistem sudah siap untuk menjalankan agen dialognya. Sistem akan melakukan \textit{loop} selamanya terhadap proses memasukkan teks. Teks yang telah dimasukkan oleh pengguna akan diprediksi maksud kalimatnya sehingga sistem dapat menentukan aksi yang tepat.

\subsubsection{spaCy}

spaCy adalah sebuah \textit{open-source library} Python digunakan untuk melakukan pengolahan teks, khususnya NLP, dalam tingkat industri \parencite{spacy2}. spaCy mendukung banyak bahasa selain MITIE, dan lebih mudah untuk menambahkan bahasa baru, cukup dengan menyediakan model bahasa yang dibutuhkan ke dalam spaCy. Versi spaCy yang digunakan adalah versi 2.0.11.

Arsitektur spaCy dapat terlihat di dalam Gambar \ref{fig:spaCy_arch} \parencite{spacy2}. Terdapat dua struktur data pusat yang berada di dalam spaCy, yaitu Doc dan Vocab. Doc berisi rangkaian dari \textit{token} dan semua anotasinya, sedangkan Vocab berisi sekumpulan tabel \textit{look-up} untuk menyediakan informasi umum ke seluruh dokumen, bertujuan untuk mencegah terjadinya duplikasi data dan memastikan bahwa fakta hanya tersimpan di dalam satu sumber.

Dengan menyimpan semua fakta di dalam satu sumber dapat menghemat penyimpanan di dalam arsitektur spaCy itu sendiri. Seperti, Doc menyimpan seluruh data-data anotasi teks, sedangkan Token dan Span hanya berupa \textit{view} yang berisi \textit{pointer} untuk merujuk kepada data yang ada di Doc.

\begin{figure}[H]
	\centering
	\includegraphics[width=0.9\textwidth, trim=2 2 2 2, clip]{resources/2/spacy_arch.pdf}
	\caption{Bagan arsitektur spaCy \parencite{spacy2}}
	\label{fig:spaCy_arch}
\end{figure}

\subsubsection{fastText}

fastText adalah \textit{library} untuk klasifikasi dan representasi teks yang dibangun oleh peneliti AI dari Facebook \parencite{joulin2017bag}. Pemrosesan yang dilakukan adalah mengubah sebuah kalimat-kalimat yang tersedia menjadi sebuah model vektor. Model vektor merepresentasikan kedekatan antara suatu kata dengan kata yang lain, memiliki nilai rentang antara 0 hingga 1. Model vektor tersebut dapat dipakai tidak hanya oleh \textit{library} fastText, namun juga dapat digunakan oleh \textit{library} luar, seperti spaCy.

Untuk pengerjaan Tugas Akhir ini, digunakan model vektor yang telah tersedia oleh fastText di dalam repositori GitHub. Model vektor yang digunakan adalah model vektor berbahasa Indonesia, yang diproses dari teks-teks yang tersedia di Wikipedia. Untuk dapat menggunakan model vektor, terlebih dahulu model vektor dikonversi menjadi model bahasa spaCy, dalam hal ini model bahasa Indonesia.

\subsection{Teori yang Diterapkan}

\subsubsection{Pengenalan Entitas dengan \textit{Conditional Random Field}}

\textit{Conditional random field} (CRF) adalah metode pembelajaran yang digunakan untuk data-data yang bersifat berurutan. CRF menggunakan fitur-fitur yang dimiliki oleh sebuah kata untuk melakukan pelatihan. Fitur yang digunakan untuk CRF dapat bervariasi sesuai kebutuhan, namun fitur dasar yang tersedia adalah kata utama dan kata sebelum kata utama. CRF biasanya digunakan untuk \textit{sequence labelling} teks.

Rasa NLU menggunakan \textit{library} Python yaitu sklearn-crfsuite. \textit{Library} ini merupakan pembungkus \textit{library} python-crfsuite yang mana menyediakan \textit{estimator} yang cocok dengan scikit-learn \parencite{sklearncrf}. Hal tersebut memudahkan penggunaan model CRF untuk pelatihan dan menyimpan model hasil latihan. Selain itu, spaCy juga menyediakan fitur yang dapat digunakan oleh CRF milik Rasa NLU, yaitu fitur POS \textit{tag} untuk tiap kata dalam kalimat.

\subsubsection{Klasifikasi Maksud Kalimat dengan \textit{Support Vector Machine}}

\textit{Support vector machine} (SVM), atau \textit{support vector network}, adalah sebuah metode pembelajaran yang memetakan vektor masukan menjadi ruang fitur dengan dimensi yang tinggi melalui pemetaan non linear yang telah dipilih sebelumnya \parencite{cortes1995support}. SVM biasanya digunakan untuk menjelaskan kegiatan klasifikasi dengan metode support vector, namun kegunaan SVM juga bisa digunakan untuk kegiatan regresi. Oleh karena itu, SVM terbagi menjadi dua, yaitu \textit{support vector classification} (SVC) untuk klasifikasi dan \textit{support vector regression} (SVR) untuk regresi \parencite{gunn1998support}.

Rasa NLU menggunakan SVM untuk melakukan klasifikasi maksud kalimat dari scikit-learn. Dengan jumlah maksud kalimat yang berjumlah lebih dari dua, maka scikit-learn akan menggunakan metode-metode untuk mengakali SVM sehingga dapat digunakan untuk kasus melakukan klasifikasi lebih dari dua kelas. Selain itu, Rasa menggunakan metode \textit{grid-search} untuk melakukan \textit{hyperparameter tuning.}

\subsection{Analisis Kekurangan Sistem}

Selama pengerjaan sistem NLU berlangsung, terdapat kekurangan-kekurangan yang dimiliki oleh alat-alat yang digunakan untuk membangun sistem. Kekurangan tersebut berkaitan dengan permasalahan proyek yang sedang dikerjakan.

%\subsubsection{Model yang Digunakan Tidak Cukup}
%
%Model vektor fastText digunakan untuk dapat membangun model bahasa Indonesia spaCy, dan digunakan dalam melakukan pengolahan teks di Rasa. Namun, model bahasa yang berhasil dibangun dari model vektor hanyalah pada bagian kosakata saja. Untuk dapat membangun model bahasa yang baik, model juga memerlukan bagian-bagian seperti \textit{named entity recognition}, \textit{parser}, serta \textit{tagger}.
%
%\textit{Named entity recognition}, atau NER, merujuk kepada pengenalan \textit{token-token} yang merupakan sebuah entitas, seperti nama oganisasi, nama tempat, tanggal, waktu, angka urutan, dan lain-lain. NER diperlukan untuk membedakan antara \textit{token} biasa dengan \textit{token} entitas. \textit{Parser} merujuk kepada \textit{dependency parsing} yaitu melihat ketergantungan sebuah token dengan token yang lain. Sebagai contoh, kalimat “Budi memakan nasi” memiliki dua ketergantungan, yaitu “Budi” memiliki ketergantungan subyek dengan “memakan”, dan “nasi” memiliki ketergantungan obyek dengan “memakan”. Terakhir, \textit{tagger} merujuk kepada \textit{POS tagging} yaitu memberikan tanda kepada token berdasarkan sifat kata dari token tersebut, misal kata kerja, kata benda, tanda baca, dan lain-lain.
%
%Terdapat beberapa korpus untuk latihan yang menyediakan komponen seperti \textit{dependency parser} dan \textit{POS tag}, salah satunya berasal dari Universal Dependencies. Universal Dependencies menyediakan korpus yang diambil dari berbagai sumber, seperti berita, artikel blog, istilah hukum, istilah medis, Wikipedia, dan lain-lain.
%
\subsubsection{Kebutuhan Ekstraksi Fitur untuk CRF}

CRF membutuhkan fitur-fitur yang telah didefinisikan sebelumnya untuk dapat melakukan latihan, seperti CRF yang diterapkan oleh Rasa membutuhkan fitur POS \textit{tag} pada sebuah kata. Fitur tersebut hanya bisa didapatkan dengan \textit{library} NLP, seperti spaCy, yang dapat melakukan latihan POS \textit{tag} berdasarkan model yang telah disediakan oleh pengguna. Oleh karena itu, sebuah kalimat masukan perlu melakukan klasifikasi dengan spaCy terlebih dahulu sebelum akhirnya diekstraksi dengan menggunakan Rasa.

Model vektor fastText digunakan untuk dapat membangun model bahasa Indonesia spaCy, dan digunakan dalam melakukan pengolahan teks di Rasa. Namun, model bahasa yang berhasil dibangun dari model vektor hanyalah pada bagian kosakata saja. Untuk dapat membangun model bahasa yang baik, model juga memerlukan bagian-bagian seperti POS \textit{tagger}. Kekurangan informasi tersebut berakibat pada klasifikasi spaCy yang menjadi tidak maksimal.

Lalu, model dari fastText hanya tersedia dalam bahasa Indonesia yang baku. Hal ini dapat menyebabkan kesulitan dalam mengenal kata-kata yang baru diucapkan, atau kata yang tidak baku seperti kata-kata dalam percakapan sehari-hari. Oleh karena itu, pendekatan CRF dari Rasa NLU tidak cocok untuk digunakan pada proyek ini.

%\subsubsection{Penggunaan SVM untuk Klasifikasi Lebih dari Dua Kelas}
%
%Rasa NLU menggunakan metode SVM dengan tambahan pengaturan grid-search dari scikit-learn untuk melakukan pelatihan klasifikasi maksud kalimat. Kelemahan yang terdapat pada SVM dalam melakukan klasifikasi adalah SVM hanya dapat melakukan klasifikasi pada dua kelas saja. SVM tidak dirancang untuk mengatasi permasalahan klasifikasi yang membutuhkan lebih dari dua kelas. Namun, scikit-learn dapat mengatasi permasalahan tersebut dengan menggunakan skema klasifikasi one-versus-one. Skema tersebut bekerja dengan cara membuat satu alat klasifikasi untuk tiap pasangan kelas yang ada. Pada saat prediksi, kelas yang menerima voting tertinggi akan terpilih [cit].
%
%Kelemahan yang dimiliki oleh skema one-versus-one adalah jumlah klasifikasi yang dilakukan, yaitu (jumlah kelas * (jumlah kelas – 1) / 2) klasifikasi, membuat skema ini memiliki kompleksitas O(jumlah kelas \^{} 2). Dengan jumlah klasifikasi yang banyak, SVM dirasa tidak efisien untuk melakukan klasifikasi dengan jumlah kelas yang banyak. Oleh karena itu, metode klasifikasi yang tepat digunakan untuk mengatasi masalah jumlah kelas lebih dari dua tersebut adalah \textit{neural network}.
%
%Keluaran dari \textit{neural network} dapat diatur sesuai dengan jumlah kelas yang tersedia pada data latihan. Fleksibilitas yang dimiliki oleh \textit{neural network} membuat metode ini dapat melakukan satu model dan satu kali latihan klasifikasi untuk dapat menghasilkan model yang bagus untuk digunakan.
    \chapter{Ajuan Sistem Baru}

\section{Rancangan Latihan Klasifikasi}

Rancangan sistem untuk tahap latihan klasifikasi disajikan dalam bagan pada Gambar \ref{fig:design_training} berikut. Proses-proses yang harus dilewati dalam tahap latihan klasifikasi dapat dijelaskan sebagai berikut:

\begin{enumerate}
	\item sistem akan memuat \textit{file} JSON yang berisi data latihan yang digunakan untuk membangun NLU,
	\item sistem mengambil data latihan tersebut lalu mengubahnya menjadi sekumpulan obyek SentenceData,
	\item sistem mengambil informasi maksud kalimat yang telah didefinisikan di dalam data latihan dan menyimpan sekumpulan maksud kalimat,
	\item sistem memecahkan sebuah kalimat menjadi sekumpulan \textit{token} kata,
	\item sistem mengekstraksi entitas yang telah didefinisikan di dalam data latihan dan mengumpulkan semua entitas yang terekstrasi,
	\item sistem menciptakan tas kata-kata dengan menggunakan metode \textit{one-hot encoding},
	\item sekumpulan maksud kalimat dan tas kata-kata digunakan sebagai masukan untuk latihan model NLU, menghasilkan model klasifikasi, dan,
	\item model klasifikasi disimpan ke dalam direktori sistem untuk digunakan pada tahap klasifikasi teks.
\end{enumerate}

\begin{figure}[H]
	\centering
	\includegraphics[width=\textwidth, trim=2 2 2 2, clip]{resources/4-design_training.pdf}
	\caption{Bagan rancangan sistem tahap latihan klasifikasi}
	\label{fig:design_training}
\end{figure}

\section{Rancangan Klasifikasi Teks}

Rancangan sistem untuk tahap klasifikasi teks disajikan dalam bagan pada Gambar \ref{fig:design_classification}. Proses-proses yang harus dilewati dalam tahap klasifikasi teks dapat dijelaskan sebagai berikut:

\begin{enumerate}
	\item sistem menerima masukan teks dari \textit{speech recognition} yang dimasukkan oleh suara pengguna,
	\item sistem memecahkan teks menjadi sekumpulan \textit{token},
	\item sistem mengenali entitas yang berada di dalam teks, kemudian mengekstraksi entitas tersebut,
	\item sistem menciptakan tas kata-kata untuk teks masukan,
	\item sistem melakukan prediksi maksud kalimat dari masukan tas kata-kata menggunakan model klasifikasi yang dihasilkan pada tahap latihan klasifikasi,
	\item maksud kalimat yang telah dihasilkan, bersama dengan kumpulan entitas hasil ekstrasi, menjadi masukan untuk memenggil aksi sistem, dan,
	\item sistem melakukan aksi kepada pengguna sebagai tanggapan dari masukan teks dari sistem ASR.
\end{enumerate}

\begin{figure}[H]
	\centering
	\includegraphics[width=\textwidth, trim=2 2 2 2, clip]{resources/4-design_classification.pdf}
	\caption{Bagan rancangan sistem tahap klasifikasi teks}
	\label{fig:design_classification}
\end{figure}

\section{Penjelasan Proses}

\subsection{Memuat Data}

Data latihan yang disediakan di dalam direktori sistem memiliki format JSON. Data tersebut berisikan teks yang akan dilatih beserta informasi maksud kalimat dan entitas yang terkandung di dalamnya jika ada. Dalam tahap latihan klasifikasi, sistem akan memuat data latihan dan memasukkan data-data menjadi obyek SentenceData. Obyek SentenceData memiliki atribut-atribut berupa teks, \textit{token-token}, dan tanda entitas tiap \textit{token}. Selain itu, proses memuat data juga menghasilkan nilai entitas yang terkandung di dalam kalimat data latihan.

\subsection{Tokenisasi}

Tokenisasi adalah proses untuk memecahkan sebuah teks kalimat menjadi sekumpulan \textit{token}. Selain kata dan angka, \textit{token} dapat juga berbentuk tanda baca. Rancangan proses tokenisasi untuk sistem NLU baru yang akan dibangun ditunjukkan dalam bagan pada Gambar \ref{fig:design_token}. Teks yang dihadapi oleh sistem berupa kalimat yang dibentuk dari huruf-huruf tanpa tanda baca, mengingat bahwa teks merupakan keluaran dari sistem ASR. Teks dipecahkan dengan menggunakan prosedur split() yang telah disediakan oleh Python. Hasil pecahan teks disimpan kembali ke dalam obyek SentenceData.

\begin{figure}[H]
	\centering
	\includegraphics[width=0.8\textwidth, trim=2 2 2 2, clip]{resources/4-design_token.pdf}
	\caption{Rancangan proses tokenisasi}
	\label{fig:design_token}
\end{figure}

\subsection{Ekstraksi Entitas}

Ekstraksi entitas adalah proses untuk mengambil entitas-entitas yang terkandung di dalam teks. Entitas tersebut akan digunakan sebagai masukan aksi yang akan dijalankan sistem NLU setelah memprediksi maksud kalimat dari teks tersebut.

Gambar \ref{fig:design_entity} menunjukkan rancangan untuk proses ekstraksi entitas dalam sistem NLU baru. Dalam tahap latihan klasifikasi, entitas-entitas yang ada di dalam teks telah didefinisikan di dalam data latihan. Sistem akan menandakan \textit{token} mana saja yang termasuk ke dalam entitas tersebut dengan menggunakan skema \textit{tag} BILUO [cit]. Penanda tersebut akan berguna untuk melakukan latihan pengenalan entitas, yang mana model hasil dari latihan tersebut disimpan ke dalam direktori sistem. Untuk proses latihan klasifikasi, semua \textit{token} yang merupakan entitas diganti menjadi satu \textit{token} yang bernilai nama entitas tersebut.

Dalam tahap klasifikasi teks, sebuah teks polos masukan dari sistem ASR akan ditebak entitas-entitas di dalamnya dengan menggunakan model latihan pengenalan entitas yang telah dijalankan sebelumnya. Setelah semua \textit{token} telah ditandai, sistem akan mengambil nilai entitas-entitas yang terkandung dan menyimpan nilai tersebut untuk digunakan setelah klasifikasi maksud teks telah dijalankan.

\begin{figure}[H]
	\centering
	\includegraphics[width=\textwidth, trim=2 2 2 2, clip]{resources/4-design_entity.pdf}
	\caption{Rancangan proses ekstraksi entitas}
	\label{fig:design_entity}
\end{figure}

Beberapa algoritma dipilih untuk mengatasi kelemahan yang terdapat pada CRF milik Rasa NLU. Pilihan algoritma yang cocok adalah sebagai berikut:

\begin{enumerate}
	\item menggunakan \textit{recurrent neural network} dari Keras,
	\item menggunakan \textit{convolutional neural network} dari Keras,
	\item menggunakan \textit{convolutional neural network} dua arah [cit],
\end{enumerate}

\subsection{Menciptakan Tas Kata-Kata}

Tas kata-kata, atau lebih dikenal dengan \textit{bag-of-words} (BOW), adalah bentuk representasi sebuah kalimat dengan menunjukkan kehadiran kata-kata dalam kosakata di dalam kalimat tersebut. Pembentukan tas kata-kata dapat dilihat pada Gambar \ref{fig:design_bow}. Tas kata-kata diciptakan dengan menggunakan teknik \textit{one-hot encoding}, yaitu merepresentasikan kosakata yang muncul di dalam sebuah kalimat ke dalam bentuk biner. Pembuatan kosakata dilakukan dengan melacak semua kata yang muncul di dalam data latihan yang telah diberikan. Untuk kata-kata yang menjadi entitas, kata tersebut diganti menjadi satu token berisi nama entitas.

\begin{figure}[H]
	\centering
	\includegraphics[width=0.5\textwidth, trim=3 3 3 3, clip]{resources/4-design_bow.pdf}
	\caption{Rancangan proses menciptakan tas kata-kata}
	\label{fig:design_bow}
\end{figure}

\subsection{Latihan Klasifikasi}

Latihan dijalankan untuk melatih sekumpulan tas kata-kata dan maksud kalimat yang terlibat, kemudian menciptakan model hasil latihan dan menyimpannya ke dalam direktori sistem. Gambar \ref{fig:design_train_class} menunjukkan bagaimana sistem melakukan latihan dan menciptakan model latihan. Sistem mendapatkan sekumpulan tas kata-kata dan label maksud kalimat secara bersamaan. Tas kata-kata menjadi masukan contoh masukan model, sedangkan label maksud kalimat menjadi masukan contoh keluaran model.

\begin{figure}[H]
	\centering
	\includegraphics[width=0.7\textwidth, trim=3 3 3 3, clip]{resources/4-design_train_class.pdf}
	\caption{Rancangan proses latihan klasifikasi}
	\label{fig:design_train_class}
\end{figure}

Untuk sistem NLU yang baru, \textit{neural network} digunakan. \textit{Neural network} diyakini dapat mengatasi kelemahan yang dimiliki oleh SVM saat melakukan klasifikasi dengan jumlah kelas lebih dari dua.

\subsection{Klasifikasi}

Proses klasifikasi dijalankan pada tahap klasifikasi teks. Gambaran mengenai proses ini terdapat pada Gambar \ref{fig:design_class_class}. Sistem terlebih dahulu memuat model latihan yang telah disimpan di dalam direktori sistem. Lalu, sistem menggunakan model tersebut untuk melakukan prediksi dari teks masukan yang telah diubah menjadi sebuah tas kata-kata.

\begin{figure}[H]
	\centering
	\includegraphics[width=0.7\textwidth, trim=2 2 2 2, clip]{resources/4-design_class_class.pdf}
	\caption{Rancangan proses klasifikasi}
	\label{fig:design_class_class}
\end{figure}

\subsection{Memanggil Aksi Sistem}

Maksud kalimat yang telah diprediksi di tahap klasifikasi teks, digunakan untuk memilih aksi yang akan dilakukan oleh sistem NLU kepada pengguna sistem. Aksi sistem ini pada umumnya adalah melakukan komunikasi REST API kepada sistem API SVARA. API SVARA akan memberikan respon berupa \textit{file} JSON kepada sistem NLU jika komunikasi sukses dijalankan. Kemudian, \textit{file} JSON yang diterima oleh sistem NLU diberikan kepada aplikasi yang dimiliki oleh pengguna secara langsung. Pengolahan \textit{file} JSON setelah berada di aplikasi pengguna diserahkan kepada aplikasi itu sendiri.

Beberapa aksi memerlukan parameter masukan untuk dapat melengkapi komunikasi dengan API SVARA. Parameter masukan didapatkan dari entitas-entitas yang telah diambil nilainya setelah tiap \textit{token} kalimat ditandai pada saat proses ekstraksi entitas.
    \chapter{Pengujian dan Implementasi}

\section{Pengujian}

\subsection{Lingkungan Pengujian}

Pengujian untuk kedua sistem pelatihan yang akan diukur dan dibandingkan kemampuannya, dilakukan dengan menggunakan komputer lokal. Spesifikasi lingkungan untuk komputer adalah sebagai berikut:

\begin{enumerate}
    \item Windows 10,
    \item Python 3.6 dari Anaconda,
    \item \textit{Library} Keras, dan,
    \item \textit{Library} scikit-learn.
\end{enumerate}

\subsecton{Hasil Pengujian}
\blindtext

\section{Implementasi}

\subsection{Lingkungan Implementasi}

Implementasi untuk sistem klasifikasi maksud kalimat dilakukan dengan menggunakan komputasi awan. Spesifikasi lingkungan untuk implementasi sistem adalah sebagai berikut:

\begin{enumerate}
    \item Microsoft Azure, dengan sistem server Linux,
    \item Python 3.7, dan,
    \item \textit{Library} Flask untuk menangani HTTP \textit{request}.
\end{enumerate}

Sebagai penunjang keberjalanan sistem dalam implementasi, \textit{speech recognition} ditangani oleh entitas di luar sistem implementasi, yaitu menggunakan Wit AI.
    \chapter{Penutup}

\section{Kesimpulan}

Dalam Tugas Akhir ini, telah dibuat sebuah metode normalisasi teks menggunakan algoritme jarak Levenshtein. Pengujian salah satu skenario dalam Tugas Akhir ini menunjukkan bahwa metode normalisasi usulan dengan menggunakan jarak Levenshtein unggul dalam akurasi dibandingkan dengan metode normalisasi \parencite{saragih2017normalisasi} yang menggunakan jarak LCS, dengan selisih persentase akurasi sebesar 8,34 persen. Setelah itu, dilakukan pengujian dengan skenario yang lain. Pengujian tersebut menunjukkan metode normalisasi usulan masih tetap unggul dengan selisih persentase akurasi sebesar 18,89 persen. Hal tersebut menunjukkan bahwa metode normalisasi ajuan lebih cocok digunakan dalam sistem \textit{voice assistant} karena algoritme jarak Levenshtein mendukung operasi penggantian karakter dibandingkan algoritme jarak LCS.

\section{Saran}

Saran-saran yang diberikan untuk penelitian selanjutnya adalah sebagai berikut:
\begin{enumerate}
    \item mencari solusi untuk kasus perbaikan kata yang tidak termasuk dalam kumpulan kata prediksi dengan jarak minimum; dan
    \item menentukan algoritme yang dapat digunakan untuk menentukan kata dalam kumpulan kata prediksi yang tepat digunakan.
\end{enumerate}
    %----------------------------------------------------------------%

    % Daftar pustaka
    \clearpage
    \pagenumbering{roman}
    \setcounter{page}{\thesavepage}
    \addcontentsline{toc}{chapter}{Daftar Pustaka}
    \printbibliography[title=Daftar Pustaka]

    % Index
    \clearpage
    \appendix
    \addtocontents{toc}{\protect\renewcommand{\protect\cftchappresnum}{Lampiran }}
    \addtocontents{toc}{\protect\renewcommand{\protect\cftchapnumwidth}{6em}}
    \renewcommand{\chaptername}{Lampiran}
    \pagenumbering{bychapter}
    	
    \chapter{\textit{Source Code} Python untuk Jarak Levenshtein}

\begin{lstlisting}
def lv_distance(str1, str2):
    str1 = " " + str1
    str2 = " " + str2
    
    len1 = len(str1)
    len2 = len(str2)
    
    matrix = [[0] * len2 for i in range(len1)]
    
    for i in range(1, len1):
        matrix[i][0] = i
    for j in range(1, len2):
        matrix[0][j] = j
        
    for j in range(1, len2):
        for i in range(1, len1):
            actv = 1
            if (str1[i] == str2[j]):
                actv = 0
            matrix[i][j] = min(matrix[i-1][j] + 1, matrix[i][j-1] + 1, matrix[i-1][j-1] + actv)
    
    return matrix[len1-1][len2-1]
\end{lstlisting}
    \chapter{\textit{Source Code} Python untuk Normalisasi Kata dengan Jarak Levenshtein}

\begin{lstlisting}
def findWord_lv(txt, kam):
    minVal = 30
    similarWords = [txt]
    
    for word, i in kam.items():
        val = lv_distance(txt, word)
        if (val < minVal):
            if (val == 0):
                return [word], val
            else:
                similarWords = [word]
                minVal = val
        elif (val == minVal):
            similarWords.append(word)
            
    return similarWords, minVal
\end{lstlisting}
    \chapter{\textit{Source Code} Python untuk Normalisasi Kata dengan Jarak Jaro-Winkler}

\begin{lstlisting}
from textdistance import jaro_winkler

def findWord_jw(txt, kam):
    maxVal = 1
    similarWords = [txt]
    
    for word, i in kam.items():
        val = jaro_winkler(txt, word)
        if (val > maxVal):
            if (val == 1):
                return [word], val
            else:
                similarWords = [word]
                maxVal = val
        elif (val == maxVal):
            similarWords.append(word)
            
    return similarWords, maxVal
\end{lstlisting}
    
\end{document}
