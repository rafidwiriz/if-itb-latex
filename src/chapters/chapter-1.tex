\chapter{Pendahuluan}

\section{Latar Belakang}

\textit{Voice assistant} (asisten suara) mulai menjadi tren yang makin berkembang. Jejaknya mulai terlihat belakangan ini dalam kehidupan manusia. Gartner menyebutkan, pada tahun 2018, 30 persen interaksi dengan sebuah teknologi melibatkan media yang mendukung percakapan dengan mesin pintar \parencite{escherich2015market}. Tren tersebut diperkirakan akan terus berlanjut hingga beberapa tahun ke depan. ComScore memperkirakan bahwa tahun 2020, 50 persen pencarian dalam internet merupakan pencarian berbasis suara \parencite{olson2016just}.

Berbicara dengan aplikasi \textit{voice assistant} dapat menciptakan pengalaman yang lebih baik untuk pengguna. Sebanyak 41 persen pengguna \textit{voice assistant} merasa bahwa berbicara dengan \textit{voice assistant} terasa seperti berbicara dengan teman atau orang lain \parencite{kleinberg2018five}, dan data tersebut diperkuat dengan data bahwa hampir 70 persen pencarian dengan menggunakan Google Assistant menggunakan bahasa alami ketimbang bahasa terstruktur yang biasa digunakan pada pencarian \textit{web} pada umumnya. Pengalaman tersebut menjadikan pengguna merasa nyaman menggunakan \textit{voice assistant} dalam kegiatan sehari-hari. Data dari Google menunjukkan 72 persen menggunakan aplikasi \textit{voice assistant} sebagai bagian dari rutinitas harian mereka \parencite{kleinberg2018five}.

\textit{Voice assistant} memiliki potensi untuk mempermudah kegiatan bagi masyarakat Indonesia. Namun, untuk dapat diterima oleh masyarakat, \textit{voice assistant} harus mengatasi sebuah keadaan, yaitu hanya sekitar 20 persen masyarakat Indonesia yang berbicara dengan menggunakan bahasa Indonesia yang baku untuk percakapan sehari-hari \parencite{naim2012kewarganegaraan}. Masyarakat Indonesia terbiasa menggunakan bahasa selain bahasa Indonesia untuk berbicara dengan kerabat mereka dalam lingkungan tidak resmi, yakni dengan menggunakan campur kode. Campur kode menurut Chaer dan Agustina \parencite{chaer1995sosiolinguistik} adalah pemakaian dua bahasa atau lebih atau dua varian dari sebuah bahasa dalam satu masyarakat tutur, yang salah satunya merupakan kode utama atau kode dasar yang digunakan memiliki fungsi dan keotomiannya, sedangkan kode-kode lain yang terlibat dalam peristiwa tutur hanya berupa serpihan-serpihan saja. \textit{Voice assistant} harus dapat mengerti bahasa campur kode tersebut.

Namun, untuk dapat mengakomodasi campur kode tersebut, terdapat masalah pada model bahasa untuk \textit{voice assistant}. Masalah tersebut adalah terdapat kata-kata campur kode yang tidak termasuk ke dalam kamus bahasa Indonesia yang baku. Sebagai contoh, kata "\textit{puter}" yang biasa diartikan sebagai "putar", kata "\textit{maen}" yang diartikan sebagai "main", dan lain-lain. Selain itu, campur kode juga melibatkan kata-kata yang berasal dari bahasa selain bahasa Indonesia, termasuk bahasa daerah dan bahasa asing. Padahal, model bahasa yang sudah tersedia menggunakan media-media resmi sebagai acuan untuk membentuk model bahasanya. Sebagai contoh, model bahasa fastText milik Facebook dibangun dan dilatih dengan acuan kepada kata-kata dalam Wikipedia, termasuk untuk bahasa Indonesia \parencite{bojanowski2016enriching}. Model bahasa fastText yang telah dilatih terdiri dari kamus dan nilai vektor untuk tiap kata di dalam kamus tersebut. Nilai vektor tiap kata berguna untuk "meletakkan" kata dalam sebuah "tempat" multi dimensi dan menentukan kata yang saling berdekatan dengan kata yang lain. Dengan model yang sudah dilatih tersebut, menambah beberapa kata ke dalam model bahasa tidak dimungkinkan.

Untuk mengatasi masalah ketiadaan kata dalam kamus, diperlukan sebuah metode berupa normalisasi kata, yaitu mengubah kata yang tidak baku menjadi kata baku dalam suatu bahasa. Terdapat sebuah penelitian mengenai pengembangan metode normalisasi teks pada suntingan Twitter bahasa Indonesia dengan menggunakan bahasa pemrograman R \parencite{saragih2017normalisasi}. Sistem tersebut mengukur jarak perbedaan kedua kata atau \textit{string}. Namun, metode tersebut hanya digunakan pada lingkungan Twitter yang mana sebagian besar kata yang disunting merupakan kata tidak baku yang disingkat, tapi tidak untuk kata tidak baku yang utuh atau tidak disingkat. Padahal, dalam \textit{voice assistant} sebuah kata diucapkan tanpa ada penyingkatan kata.

Oleh karena itu, diperlukan sebuah metode normalisasi teks yang dapat digunakan dalam lingkungan kata yang tidak disingkat. Dengan begitu, metode normalisasi tersebut dapat digunakan untuk sistem \textit{voice assistant} untuk bahasa Indonesia.

\section{Rumusan Masalah}

Dari latar belakang yang telah dijabarkan, masalah yang akan diselesaikan dalam pengerjaan Tugas Akhir ini adalah "bagaimana cara membangun metode normalisasi teks yang dapat digunakan untuk kata tidak baku yang tidak disingkat?"

\section{Tujuan}

Berdasarkan masalah yang telah dirumuskan sebelumnya, Tugas Akhir ini dikerjakan dengan tujuan untuk membangun fitur normalisasi teks yang dapat digunakan untuk kata tidak baku yang tidak disingkat.

\section{Batasan Masalah}

Batasan-batasan yang dterapkan dalam pengerjaan Tugas Akhir adalah permasalahan kata tidak baku yang diselesaikan hanya untuk kata-kata yang tidak disingkat.

\section{Metodologi}

Metodologi yang digunakan dalam perancangan metode nromalisasi sistem \textit{voice assistant} dalam pengerjaan Tugas Akhir ini adalah sebagai berikut:

\begin{enumerate}
	\item melakukan analisis terkait kebutuhan sistem \textit{voice assistant};
	\item melakukan studi literatur mengenai algoritme dan metode normalisasi yang akan digunakan sebagai pembanding;
	\item merancang usulan metode normalisasi;
	\item melakukan pengembangan metode normalisasi usulan dan metode normalisasi pembanding; dan
	\item melakukan evaluasi dan perbandingan antara metode normalisasi pembanding dan metode normalisasi usulan. 
\end{enumerate}

\section{Sistematika Pembahasan}

Laporan pengerjaan Tugas Akhir ini disusun dengan sistematika pembahasan sebagai berikut:

\begin{enumerate}[label=Bab \arabic*,itemindent=*]
	\item Pendahuluan\\
	Bab Pendahuluan membahas latar belakang, tujuan, batasan masalah, dan metodologi dari pengerjaan Tugas Akhir, serta sistematika penulisan laporan Tugas Akhir ini.
	\item Tinjauan Pustaka\\
	Bab Tinjauan Pustaka membahas teori-teori yang digunakan untuk menyelesaikan rumusan masalah serta meninjau penelitian yang telah dilakukan sebelumnya.
	\item Rancangan Normalisasi dengan Jarak Levenshteinn\\
	Bab ini membahas rancangan dari penelitian sebelumnya dalam bab Tinjauan Pustaka, rancangan yang diusulkan untuk mengatasi rumusan masalah, serta perbandingan antara kedua rancangan tersebut.
	\item Pengujian\\
	Bab Pengujian membahas lingkungan yang digunakan untuk pengujian, penjelasan data uji, skenario pengujian yang dijalankan, dan hasil pengujian dari rancangan yang telah dikembangkan.
	\item Penutup\\
	Bab Penutup berisi kesimpulan dari hasil pengujian rancangan yang telah dikembangkan serta saran yang dapat digunakan untuk penelitian dan pengembangan kedepan.
\end{enumerate}
