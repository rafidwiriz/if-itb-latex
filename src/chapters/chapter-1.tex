\chapter{Pendahuluan}

\section{Latar Belakang}

\textit{Voice assistant} mulai menjadi tren yang makin berkembang. Jejaknya mulai terlihat belakangan ini dalam kehidupan manusia. Gartner menyebutkan, pada tahun 2018, 30 persen interaksi dengan sebuah teknologi melibatkan media yang mendukung percakapan dengan mesin pintar \parencite{escherich2015market}. Tren tersebut diperkirakan akan terus berlanjut hingga beberapa tahun kedepan. ComScore memperkirakan bahwa tahun 2020, 50 persen pencarian dalam internet merupakan pencarian berbasis suara.

Berbicara dengan aplikasi \textit{voice assistant} dapat menciptakan pengalaman yang lebih baik untuk pengguna. 41 persen pengguna \textit{voice assistant} merasa bahwa berbicara dengan \textit{voice assistant} terasa seperti berbicara dengan teman atau orang lain \parencite{kleinberg2018five}. Dan data tersebut diperkuat dengan data bahwa hampir 70 persen pencarian dengan menggunakan Google Assistant menggunakan bahasa alami ketimbang bahasa terstruktur yang biasa diguakan pada pencarian web pada umumnya. Pengalaman tersebut menjadikan pengguna merasa nyaman menggunakan \textit{voice assistant} dalam kegiatan sehari-hari. Data dari Google menunjukkan 72 persen menggunakan aplikasi \textit{voice assistant} sebagai bagian dari rutinitas harian mereka \parencite{kleinberg2018five}.

Dengan keuntungan yang bagus, \textit{voice assistant} memiliki potensi untuk mempermudah kegiatan bagi masyarakat Indonesia. Namun, untuk dapat diterima oleh masyarakat, \textit{voice assistant} harus mengatasi sebuah keadaan yang mana hampir 20 persen dari seluruh masyarakat Indonesia yang berbicara dengan menggunakan bahasa Indonesia untuk percakapan sehari-hari \parencite{naim2012kewarganegaraan}. Masayarakat Indonesia terbiasa menggunakan bahasa selain bahasa Indonesia untuk berbicara dengan kerabat mereka dalam lingkungan tidak resmi, lebih banyak menggunakan bahasa Indonesia dengan campur kode. Campur kode menurut Chaer dan Agustina \parencite{chaer1995sosiolinguistik} adalah pemakaian dua bahasa atau lebih atau dua varian dari sebuah bahasa dalam satu masyarakat tutur, dimana salah satu merupakan kode utama atau kode dasar yang digunakan yang memiliki fungsi dan keotomiannya, sedangkan kode-kode lain yang terlibat dalam peristiwa tutur hanya berupa serpihan-serpihan saja. Dengan begitu, \textit{voice assistant} harus dapat mengerti bahasa campur kode tersebut dalam hal menambah pengalaman untuk para pengguna berupa masyarakat Indonesia.

Namun, masalah terjadi pada model bahasa untuk \textit{voice assistant}. Masalah yang terjadi adalah terdapat kata-kata campur kode yang tidak termasuk ke dalam kamus bahasa Indonesia yang baku. Sebagai contoh, kata "\textit{puter}" yang biasa diartikan sebagai kata "putar", "\textit{maen}" yang diartikan sebagai "main", dan lain-lain. Selain itu, campur kode juga melibatkan kata-kata yang berasal dari bahasa selain bahasa Indonesia, termasuk bahasa daerah dan bahasa asing. Padahal, model bahasa yang sudah tersedia menggunakan media-media resmi sebagai acuan untuk membentuk model bahasanya. Sebagai contoh, model bahasa fastText milik Facebook dibangun dan dilatih dengan acuan kepada kata-kata dalam Wikipedia, termasuk untuk bahasa Indonesia \parencite{bojanowski2016enriching}. Model bahasa fastText yang telah dilatih terdiri dari kamus dan nilai vektor untuk tiap kata di dalam kamus tersebut. Nilai vektor tiap kata berguna untuk "meletakkan" kata dalam sebuah "tempat" multi dimensi dan menentukan kata yang saling berdekatan dengan kata yang lain. Dengan model yang sudah dilatih tersebut, menambah beberapa kata ke dalam model bahasa tidak dimungkinkan.

Untuk mengatasi masalah ketiadaan kata dalam kamus, diperlukan sebuah metode berupa normalisasi kata, yaitu mengubah kata yang tidak baku menjadi kata baku dalam suatu bahasa. Terdapat sebuah penelitian mengenai pembuatan sistem normalisasi teks pada suntingan Twitter bahasa Indonesia dengan menggunakan bahasa pemrograman R \parencite{saragih2017normalisasi}. Sistem tersebut mengukur jarak perbedaan kedua kata atau \textit{string} dengan menggunakan fungsi \textit{stringdist} dengan 10 variasi metode penghitungan jarak yang sudah disediakan. Namun, sistem tersebut memiliki beberapa kelemahan, seperti sistem harus mendefinisikan terlebih dahulu kamus kata-kata yang tidak baku beserta perbaikan kata bakunya secara manual, termasuk kata tidak baku yang hanya berjarak beberapa huruf dengan kata baku yang mirip. Pendefinisian tersebut menambah beban pekerjaan pengurus sistem, terlebih jika terdapat kata tidak baku yang baru. Lalu, kamus perbaikan kata tidak baku tersebut diletakkan sebelum pencarian dalam kamus kata baku menggunakan \textit{stringdist} sehingga peran terbesar dalam sistem tersebut ada pada kamus perbaikan kata tidak baku.

Untuk menghindari penambahan kata kamus untuk normalisasi, dibutuhkan pendekatan berupa normalisasi dengan menggunakan kamus yang sudah tersedia, kemudian membuat nilai vektor untuk tiap kata dalam kamus dengan pelatihan sehingga sistem \textit{voice assistant} dapat memperkirakan kata masukan yang tidak terdapat dalam kamus. Selain itu, diperlukan penanganan kasus berupa kata yang sama sekali tidak memiliki kemiripan dengan kata yang ada di dalam kamus.

\section{Rumusan Masalah}

Dari latar belakang yang telah dijabarkan, masalah yang akan diselesaikan dalam pengerjaan Tugas Akhir ini adalah "bagaimana cara membangun fitur normalisasi untuk mengatasi kelemahan sistem yang tidak dapat memperkirakan kata yang tidak ada dalam kamus?"

\section{Tujuan}

Berdasarkan masalah yang telah dirumuskan sebelumnya, Tugas Akhir ini dikerjakan dengan tujuan untuk membangun fitur normalisasi untuk mengatasi kelemahan sistem yang tidak dapat memperkirakan kata yang tidak ada dalam kamus.

\section{Batasan Masalah}

Batasan masalah dalam pengerjaan Tugas Akhir ini hanya terletak pada bagian klasifikasi teks. Pengerjaan Tugas Akhir tidak terlibat pada sistem suara menjadi teks. Selain itu, batasan masalah juga terletak pada kata-kata yang diucapkan secara lisan saja sehingga tidak ada kata yang disingkat.

\section{Metodologi}

Metodologi yang digunakan dalam perancangan fitur nromalisasi sistem \textit{voice assistant} dalam pengerjaan Tugas Akhir ini adalah sebagai berikut:

\begin{enumerate}
	\item melakukan analisis terkait kebutuhan sistem \textit{voice assistant},
	\item melakukan studi literatur mengenai metode-metode yang akan digunakan untuk fitur normalisasi,
	\item melakukan pengembangan sistem untuk mengimplementasi fitur dengan metode yang telah ditentukan, dan,
	\item melakukan evaluasi sistem setelah fitur diimplementasi. 
\end{enumerate}

\section{Sistematika Pembahasan}

Laporan pengerjaan Tugas Akhir ini disusun dengan sistematika sebagai berikut:

\begin{enumerate}[label=Bab \arabic*,itemindent=*]
	\item Pendahuluan\\
	Bab ini membahas latar belakang, tujuan, batasan masalah, metodologi, dan sistematika penulisan.
	\item Dasar Teori\\
	Bab ini membahas teori yang digunakan untuk menyelesaikan rumusan masalah.
	\item Metodologi dan Rancangan\\
	Bab ini membahas metodologi pengerjaan Tugas Akhir dan rancangan yang diajukan untuk mengatasi rumusan masalah.
	\item Implementasi dan Pengujian\\
	Bab ini membahas implementasi dan hasil pengujian rancangan baru yang telah dibangun beserta penjelasan spesifikasi lingkungan yang digunakan.
	\item Penutup\\
	Bab ini berisi kesimpulan dari rancangan yang telah diuji serta saran yang dapat digunakan untuk pengembangan kedepan.
\end{enumerate}
