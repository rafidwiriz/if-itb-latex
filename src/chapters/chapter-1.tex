\chapter{Pendahuluan}

\section{Latar Belakang}

Kebanyakan manusia dapat mengucapkan kalimat lebih cepat dibandingkan dengan mengetikkannya. Kecepatan rata-rata bicara seseorang dalam percakapan biasa berada pada rentang 120 hingga 150 kata per menit \parencite{barnard2018average}. Sedangkan, kecepatan rata-rata seseorang dalam mengetik yaitu 37 kata per menit untuk wanita dan 44 kata per menit untuk pria \parencite{ratatype}.

Berbicara dengan aplikasi \textit{voice assistant} dapat menciptakan pengalaman yang lebih baik untuk pengguna. 41 persen pengguna \textit{voice assistant} merasa bahwa berbicara dengan \textit{voice assistant} terasa seperti berbicara dengan teman atau orang lain \parencite{kleinberg2018five}. Lalu, 72 persen menggunakan aplikasi \textit{voice assistant} sebagai bagian dari rutinitas harian mereka. Terakhir, \textit{speaker} yang dilengkapi \textit{voice assistant} mempermudah pengguna untuk melakukan beberapa pekerjaan sekaligus.

Pencarian dengan menggunakan suara semakin populer saat ini. Pada tahun 2020, diprediksi bahwa 50 persen dari seluruh pencarian merupakan pencarian yang menggunakan suara \parencite{}. Dan pada tahun yang sama, 30 persen pencarian dilakukan tanpa perlu menggunakan layar \parencite{}.

Suara atau audio menjadi media yang mulai kembali populer saat ini, termasuk diantaranya adalah \textit{digital audio}. Dengan adanya media digital audio, seluruh orang dapat menghibur diri sendiri sambil melakukan kegiatan sehari-hari. Menurut data \textit{survey} dari Activate pada Juni 2015, masyarakat AS menghabiskan 39 menit per hari, atau sekitar 19 persen dari waktu total per hari, pada mobile app untuk kategori audio [1]. Dan dari sumber data yang sama pada tahun 2014, tipikal masyarakat AS berumur 13 tahun ke atas menghabiskan waktu 4 jam 5 menit per hari untuk mendengarkan audio [ACT15].

Dengan melihat data tersebut, dapat dilihat bahwa bisnis audio dapat berkembang dengan konstan, khusunya dalam bidang \textit{internet radio}. Data dari eMarketer menunjukkan peningkatan pendengar bulanan \textit{internet radio} di AS yang menembus angka 50 persen dari jumlah pengguna internet sejak tahun 2012 [---13]. Dengan begitu, bisnis internet radio bisa menjanjikan keuntungan yang berlimpah.

Namun, terdapat keterbatasan yang dihadapi oleh \textit{internet radio}, salah satunya adalah lamanya interaksi yang diperlukan untuk mencapai suatu tujuan tertentu, seperti mencari radio yang pas atau lagu yang ingin didengarkan. Interaksi yang dilakukan seperti mengetik dan melakukan klik ke berbagai tempat. Untuk mengatasi keterbatasan tersebut, telah ditemukan sebuah solusi yaitu menerapkan \textit{voice assistant} untuk aplikasi, seperti apa yang sudah dilakukan oleh Google dan Apple. Hadirnya voice assistant terbukti dapat membantu interaksi manusia dengan teknologi, bahkan menurut Gartner, pada tahun 2018, 30 persen interaksi dengan teknologi akan dilakukan dengan media “\textit{conversation}” (percakapan) dengan mesin pintar [ESC16].

Dengan melihat peluang untuk menghilangkan keterbatasan tersebut, Zamrud Technology mulai melakukan rencana inovasi untuk salah satu aplikasi \textit{internet radio} mereka, yaitu SVARA. SVARA adalah aplikasi internet radio yang menyediakan layanan berupa \textit{music streaming, radio streaming}, dan juga konten-konten khusus pada radio tertentu. SVARA juga menyediakan tempat bagi pemusik lokal untuk memasukkan karya mereka. Dengan penerapan \textit{voice assistant} dalam SVARA, diharapkan penggunaan SVARA menjadi lebih mudah daripada sebelumnya.

\section{Rumusan Masalah}

Masalah-masalah yang akan diselesaikan dalam pengerjaan Tugas Akhir ini adalah sebagai berikut:

\begin{enumerate}
    \item menentukan jenis-jenis model pembelajaran yag digunakan dalam \textit{voice assistant},
    \item merancang model baru pembelajaran \textit{voice assistant},
    \item evaluasi model pembelajaran yang telah diterapkan.
\end{enumerate}

\section{Tujuan}

Tugas Akhir ini dikerjakan untuk membangun sebuah sistem \textit{voice assistant} baru. Sistem \textit{voice assistant} ini akan mengatasi beberapa permasalahan algoritma yang digunakan oleh sistem \textit{voice assistant} yang sudah ada.

\section{Batasan Masalah}

Perancangan dan implementasi dari sistem \textit{voice assistant} baru hanya terbatas pada pemrosesan teks yang dihasilkan oleh suara menjadi perintah yang telah didefinisikan oleh sistem dari awal.

\section{Metodologi}

Metodologi yang digunakan dalam perancangan fitur \textit{voice assistant} dalam pengerjaan Tugas Akhir ini adalah sebagai berikut:

\begin{enumerate}
	\item melakukan studi literatur mengenai pemrosesan teks menjadi sebuah makna,
	\item melakukan perancangan model tepat guna untuk fitur \textit{voice assistant},
	\item melakukan pengembangan fitur \textit{voice assistant} dengan merujuk kepada model yang telah dirancang, dan,
	\item melakukan implementasi fitur yang telah dikembangkan. 
\end{enumerate}

\section{Sistematika Pembahasan}

\begin{enumerate}[label=Bab \arabic*,itemindent=*]
	\item Pendahuluan\\
	Bab ini membahas latar belakang, tujuan, batasan masalah, metodologi, dan sistematika penulisan.
	\item Dasar Teori\\
	Bab ini membahas teori yang digunakan untuk merancang fitur \textit{voice assistant}.
	\item Analisis Produk\\
	Bab ini membahas rancangan fitur \textit{voice assistant} yang akan diimplementasikan pada aplikasi SVARA.
	\item Ajuan Sistem Baru\\
	Bab ini membahas implementasi dan pengujian fitur \textit{voice assistant} yang telah dikerjakan.
	\item Kesimpulan dan Saran\\
	Bab ini berisi kesimpulan yang didapatkan dari implementasi fitur serta saran yang dapat digunakan untuk memperbaiki fitur kedepan.
\end{enumerate}
