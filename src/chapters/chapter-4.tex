\chapter{Pengujian dan Implementasi}

\section{Pengujian}

\subsection{Lingkungan Pengujian}

Pengujian untuk kedua sistem SLU bagian pelatihan, Rasa NLU dan sistem baru, yang akan diukur dan dibandingkan kemampuannya, dilakukan dengan menggunakan komputer lokal. Spesifikasi lingkungan untuk komputer adalah sebagai berikut:

\begin{enumerate}
    \item Windows 10,
    \item Python 3.6 dari Anaconda,
    \item \textit{Library} Keras, dan,
    \item \textit{Library} scikit-learn.
\end{enumerate}

Sebelum melakukan pengujian, kedua sistem terlebih dahulu melakukan pelatihan pada sebuah \textit{file} JSON yang disiapkan sebagai data latih. \textit{File} JSON tersebut berisi kumpulan kalimat-kalimat perintah berbentuk teks, beserta dengan maksud kalimat tersebut dan entitas-entitas yang terkandung di dalamnya.

\subsection{Hasil Pengujian}
\blindtext

\section{Implementasi}

\subsection{Lingkungan Implementasi}

Implementasi untuk sistem SLU bagian klasifikasi dilakukan dengan menggunakan komputasi awan. Spesifikasi lingkungan untuk implementasi sistem adalah sebagai berikut:

\begin{enumerate}
    \item Microsoft Azure, dengan sistem server Linux,
    \item Python 3.7, dan,
    \item \textit{Library} Flask untuk menangani HTTP \textit{request}.
\end{enumerate}

Sebagai penunjang keberjalanan sistem dalam implementasi, \textit{speech recognition} ditangani oleh entitas di luar sistem implementasi, yaitu menggunakan Wit AI. Dalam implementasi, Wit AI hanya digunakan untuk melakukan \textit{speech recognition} saja, dan tidak digunakan untuk melakukan pemrosesan SLU.