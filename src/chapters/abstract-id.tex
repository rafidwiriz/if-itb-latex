\clearpage
\chapter*{ABSTRAK}
\addcontentsline{toc}{chapter}{Abstrak}

%taruh abstrak bahasa indonesia di sini
Teknologi \textit{voice assistant} (asisten suara) mulai berkembang pesat saat ini. Penggunaannya sudah mulai merambah kepada penggunaan sehari-hari. Namun, \textit{voice assistant} masih terbatas pada penggunaan bahasa percakapan yang baku. Sementara itu, masyarakat Indonesia terbiasa mengucapkan bahasa tidak baku dalam percakapan sehari-hari. Pengerjaan Tugas Akhir ini mencakup solusi untuk mengatasi permasalahan \textit{voice assistant} dengan kata yang tidak baku atau tidak termasuk dalam kamus kata baku. Pendekatan yang digunakan sebagai solusi adalah melakukan normalisasi teks menggunakan jarak Levenshtein dan jarak Jaro-Winkler. Hasil pengujian menunjukkan bahwa normalisasi dengan jarak Levenshtein mengungguli normalisasi dengan jarak LCS dengan selisih akurasi masing-masing sebesar 8,34 persen.

\vspace{15mm}
Kata kunci: \textit{voice assistant}, kamus, kata tidak baku, normalisasi, jarak Levenshtein, jarak Jaro-Winkler.
\clearpage