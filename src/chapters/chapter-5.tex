\chapter{Penutup}

\section{Kesimpulan}

Pengerjaan Tugas Akhir menunjukkan bahwa terdapat cara yang digunakan untuk mengatasi kata yang tidak tersedia pada kosakata sebuah bahasa. Cara yang digunakan adalah dengan menerapkan kategori "kata tidak diketahui" dan jarak Levenshtein untuk mencari kata yang mirip dengan yang ada di dalam kosakata. Diperlukan keputusan untuk menentukan seberapa jauh perbandingan sebuah kata dengan kata dalam kosakata sehingga diputuskan bahwa kata tersebut termasuk ke dalam "kata tidak diketahui".

\section{Saran}

Fungsi kosakata telah berjalan dengan baik, namun masih diperlukan perbaikan agar fungsi dapat menjadi lebih baik. Fungsi kosakata hanya dapat mencari kata yang mirip secara huruf-huruf yang dimiliki. Perlu dikembangkan lebih lanjut untuk fungsi kosakata sehingga fungsi kosakata diharapkan dapat mengetahui kata yang mirip secara makna.