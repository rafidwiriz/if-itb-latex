\chapter{Penutup}

\section{Kesimpulan}

Dalam Tugas Akhir ini, telah dibuat sebuah metode normalisasi teks menggunakan algoritme jarak Levenshtein dan jarak Jaro-Winkler. Pengujian salah satu skenario dalam Tugas Akhir ini menunjukkan bahwa metode normalisasi dengan menggunakan jarak Levenshtein unggul dalam akurasi dibandingkan dengan metode normalisasi \parencite{saragih2017normalisasi} yang menggunakan jarak LCS, dengan selisih persentase akurasi masing-masing sebesar 8,34. Setelah itu, dilakukan pengujian dengan skenario yang lain. Pengujian tersebut menunjukkan metode normalisasi dengan jarak Levenshtein masih tetap unggul dengan selisih persentase akurasi dengan metode normalisasi \parencite{saragih2017normalisasi} sebesar 18,89 persen. Hal tersebut menunjukkan bahwa metode normalisasi dengan jarak Levenshtein lebih cocok digunakan dalam sistem \textit{voice assistant} karena algoritme jarak Levenshtein mendukung operasi penggantian karakter dibandingkan algoritme jarak LCS dan algoritme jarak Jaro-Winkler.

\section{Saran}

Saran-saran yang diberikan untuk penelitian selanjutnya adalah sebagai berikut:
\begin{enumerate}
    \item mencari solusi untuk kasus perbaikan kata yang tidak termasuk dalam kumpulan kata prediksi dengan jarak minimum; dan
    \item menentukan algoritme yang dapat digunakan untuk menentukan kata dalam kumpulan kata prediksi yang tepat digunakan.
\end{enumerate}